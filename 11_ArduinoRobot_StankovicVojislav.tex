% !TEX encoding = UTF-8 Unicode

\documentclass[a4paper]{article}

\usepackage[T2A]{fontenc} % enable Cyrillic fonts
\usepackage[utf8x,utf8]{inputenc} % make weird characters work
\usepackage[serbian]{babel}
%\usepackage[english,serbianc]{babel}
\usepackage{amssymb}

\usepackage{color}
\usepackage{url}
\usepackage[unicode]{hyperref}
\hypersetup{colorlinks,citecolor=green,filecolor=green,linkcolor=blue,urlcolor=blue}

\newcommand{\odgovor}[1]{\textcolor{blue}{#1}}

\begin{document}
	
	\title{Arduino automobil na daljinsko upravljanje\\ \small{Doža Daniel, Dorde Milićević}}
	
	%%%%%%%%%%%%%%%%%%%%%%%%%%%%%%%%%%%%%%%%%%%%%%%%%%%%%%%%%%%%%%%%%%%%%%%%%%%%%%%%%%%%%%%%%%
	\author{Recenzija: Vojislav Stanković}
	%ime autora recenzije neće biti predato autorima seminarskog rada
	%%%%%%%%%%%%%%%%%%%%%%%%%%%%%%%%%%%%%%%%%%%%%%%%%%%%%%%%%%%%%%%%%%%%%%%%%%%%%%%%%%%%%%%%%%
	
	
	\maketitle
	
	%Recenziju predajete u tex obliku, budite uredni!
	%Pod komentarima su data objašnjenja za svaku navedenu stavku.
	
	%!!!ne brisite narednu liniju!!!
	%pocetak teksta koji se predaje recenzentima
	
	
	\section{O čemu rad govori?}
	% Напишете један кратак пасус у којим ћете својим речима препричати суштину рада (и тиме показати да сте рад пажљиво прочитали и разумели). Обим од 200 до 400 карактера.
	Rad govori o tome kako je realizovan automobil na daljinsko upravljanje pomoću Arduino Robot platforme.
	Opisana je sama Arduino platforma, šta čini automobil kao i kako se njime upravlja.
	Takođe je napomenuto koje su prepreke postojale pri razvoju same platforme i dati su predlozi za prevazilaženje istih.
	Napisan je kod koji simulira sam automobil.
	
	\section{Krupne primedbe i sugestije}
	% Напишете своја запажања и конструктивне идеје шта у раду недостаје и шта би требало да се промени-измени-дода-одузме да би рад био квалитетнији.
	Fali programerski aspekt. Treba objasniti kako se razvija aplikacija za automobil na samoj platformi.
	Zaključak treba popraviti.
	
	\section{Sitne primedbe}
	% Напишете своја запажања на тему штампарских-стилских-језичких грешки
	Ima nekoliko grešaka u kucanju:
	Strana 2, deseti red - priširivost (treba proširivost).
	Strana 3, poslednja rečenica - korićena (treba korišćena).
	
	\section{Provera sadržajnosti i forme seminarskog rada}
	% Oдговорите на следећа питања --- уз сваки одговор дати и образложење
	
	\begin{enumerate}
		\item Da li rad dobro odgovara na zadatu temu?\\ Rad lepo odgovara na temu. Opisana je platforma, automobil, kontrole i problemi u razvoju i napisan je k\^ od.
		\item Da li je nešto važno propušteno?\\ Fali programerski aspekt. Kako se razvijaju aplikacije na samoj platformi? Dat je k\^ od ali nije objašnjeno.
		\item Da li ima suštinskih grešaka i propusta?\\Rad ima suštinu, tema nije promašena.
		\item Da li je naslov rada dobro izabran?\\ Naslov je veoma dobar, prosto mami na čitanje.
		\item Da li sažetak sadrži prave podatke o radu?\\ Sažetak je veoma dobar, daje uvid u to o čemu rad govori.
		\item Da li je rad lak-težak za čitanje?\\ Rad se uglavnom čita u jednom dahu, jedino se usporava kod tehničkih detalja samog automobila, ali takva je priroda rada.
		\item Da li je za razumevanje teksta potrebno predznanje i u kolikoj meri?\\ Nije potrebno veće znanje od znanja iz srednje škole.
		\item Da li je u radu navedena odgovarajuća literatura?\\ Nisu navedeni nikakvi tekstovi na temu osim pristrasnih tekstova same kompanije Arduino (njihov sajt je glavna referenca).
		\item Da li su u radu reference korektno navedene?\\ Reference su adekvatne.
		\item Da li je struktura rada adekvatna?\\ Struktura je adekvatna, sa finim uvodom, dobrom razradom, i malo slabijim zaključkom (nije ništa o temi napisano u zaključku, već kako su napisali k\^ od).
		\item Da li rad sadrži sve elemente propisane uslovom seminarskog rada (slike, tabele, broj strana...)?\\ Rad sadrži sve propisane elemente. Broj strana je 5, ali ima k\^ od koji je trebalo napisati tako da je u redu.
		\item Da li su slike i tabele funkcionalne i adekvatne?\\ Slike i tabele su funkcionalne i adekvatne.
	\end{enumerate}
	
	\section{Ocenite sebe}
	% Napišite koliko ste upućeni u oblast koju recenzirate: 
	% a) ekspert u datoj oblasti
	% b) veoma upućeni u oblast
	% c) srednje upućeni
	% d) malo upućeni 
	% e) skoro neupućeni
	% f) potpuno neupućeni
	% Obrazložite svoju odluku
	e) Skoro neupućen\\
	Pročitao sam jedan ili dva veoma kratka članka o ovoj temi ranije, ali ne dovoljno za potpuno razumevanje same platforme i svih detalja.
	
	
	%kraj teksta koji se predaje recenzentima
	%!!!ne brisite prethodnu liniju!!!
	
	%%%%%%%%%%%%%%%%%%%%%%%%%%%%%%%%%%%%%%%%%%%%%%%%%%%%%%%%%%%%%%%%%%%%%%%%%%%%%%%%%%%%%%%%%%
	\section{Poverljivi komentari}
	% Poverljivi komentari neće biti prosleđeni autorima seminarskog rada.
	% Ukoliko nemate poverljivih komentara, ovaj deo može da ostane prazan.
	%%%%%%%%%%%%%%%%%%%%%%%%%%%%%%%%%%%%%%%%%%%%%%%%%%%%%%%%%%%%%%%%%%%%%%%%%%%%%%%%%%%%%%%%%%
	Kolege su veoma dobro uradile posao. Mene kao programera bi više zanimalo kako da čeprkam po tom automobilu stoga bi mogli da ubace jedan deo koji opisuje sam razvoj aplikacija, ali možda grešim i to nije bila tema, i ispravite me ako je tako.
	
\end{document}


