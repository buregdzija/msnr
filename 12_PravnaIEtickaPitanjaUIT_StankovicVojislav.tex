% !TEX encoding = UTF-8 Unicode

\documentclass[a4paper]{article}

\usepackage[T2A]{fontenc} % enable Cyrillic fonts
\usepackage[utf8x,utf8]{inputenc} % make weird characters work
\usepackage[serbian]{babel}
%\usepackage[english,serbianc]{babel}
\usepackage{amssymb}

\usepackage{color}
\usepackage{url}
\usepackage[unicode]{hyperref}
\hypersetup{colorlinks,citecolor=green,filecolor=green,linkcolor=blue,urlcolor=blue}

\newcommand{\odgovor}[1]{\textcolor{blue}{#1}}

\begin{document}

\title{Pravne i etičke obaveze u svetu
informacionih tehnologija\\ \small{Nikola Dokmanović, Milos Krsmanović}}

%%%%%%%%%%%%%%%%%%%%%%%%%%%%%%%%%%%%%%%%%%%%%%%%%%%%%%%%%%%%%%%%%%%%%%%%%%%%%%%%%%%%%%%%%%
\author{Recenzija: Vojislav Stanković}
%ime autora recenzije neće biti predato autorima seminarskog rada
%%%%%%%%%%%%%%%%%%%%%%%%%%%%%%%%%%%%%%%%%%%%%%%%%%%%%%%%%%%%%%%%%%%%%%%%%%%%%%%%%%%%%%%%%%


\maketitle

%Recenziju predajete u tex obliku, budite uredni!
%Pod komentarima su data objašnjenja za svaku navedenu stavku.

%!!!ne brisite narednu liniju!!!
%pocetak teksta koji se predaje recenzentima


\section{O čemu rad govori?}
% Напишете један кратак пасус у којим ћете својим речима препричати суштину рада (и тиме показати да сте рад пажљиво прочитали и разумели). Обим од 200 до 400 карактера.
Rad objašnjava šta je etika, naglašava značaj etike u IT sektoru. Daje sugestije kako treba postupati u skladu sa etikom kao IT stručnjak, ali i svakodnevni korisnik računara, i objašnjava značaj obrazovanja IT stručnjaka o etici.\\
Potom je objašnjen značaj donošenja zakona vezanih za IT sektor (internet kupovina, privatnost podataka, intelektualna svojina), ali i koje probleme donosi neredovno ažuriranje istih zakona ili njihovo potpuno odsustvo.

\section{Krupne primedbe i sugestije}
% Напишете своја запажања и конструктивне идеје шта у раду недостаје и шта би требало да се промени-измени-дода-одузме да би рад био квалитетнији.
Naslov mi se čini predugačkim. Možda bi bilo bolje da glasi: $"$Pravne i etičke obaveze u IT svetu$"$. Verujem da ljudi lakše prepoznaju $"$IT$"$ kao oznaku za svet računarstva nego $"$informacione tehnologije$"$.

\section{Sitne primedbe}
% Напишете своја запажања на тему штампарских-стилских-језичких грешки
Nemam nikakvih sitnih primedbi primedbi.

\section{Provera sadržajnosti i forme seminarskog rada}
% Oдговорите на следећа питања --- уз сваки одговор дати и образложење

\begin{enumerate}
\item Da li rad dobro odgovara na zadatu temu?\\ Rad odgovara na sva pitanja i to veoma dobro.
\item Da li je nešto važno propušteno?\\ Ne bih rekao da je išta propušteno.
\item Da li ima suštinskih grešaka i propusta?\\ Suštinskih grešaka nema. Tema nije promašena, rad ima suštinu.
\item Da li je naslov rada dobro izabran?\\ Naslov mi se čini predugačkim.
\item Da li sažetak sadrži prave podatke o radu?\\ Sažetak je lep i fino uvodi u priču. Daje veoma lep uvid u to šta se nalazi u radu.
\item Da li je rad lak-težak za čitanje?\\ Rad se veoma lako čita, i veoma je zanimljiv.
\item Da li je za razumevanje teksta potrebno predznanje i u kolikoj meri?\\ Nije potrebno nikakvo pravno znanje, niti stručno znanje iz oblasti IT.
\item Da li je u radu navedena odgovarajuća literatura?\\ Literatura je odgovarajuća.
\item Da li su u radu reference korektno navedene?\\ Reference su korektno navedene.
\item Da li je struktura rada adekvatna?\\ Ima sažetak, fin uvod, odličnu razradu i odličan zaključak.
\item Da li rad sadrži sve elemente propisane uslovom seminarskog rada (slike, tabele, broj strana...)?\\ Rad sadrži sve elemente propisane uslovom seminarskog rada.
\item Da li su slike i tabele funkcionalne i adekvatne?\\ Slike i tabele su funkcionalne i adekvatne.
\end{enumerate}

\section{Ocenite sebe}
% Napišite koliko ste upućeni u oblast koju recenzirate: 
% a) ekspert u datoj oblasti
% b) veoma upućeni u oblast
% c) srednje upućeni
% d) malo upućeni 
% e) skoro neupućeni
% f) potpuno neupućeni
% Obrazložite svoju odluku
b) Veoma sam upućen u oblast, pogotovo o privatnosti podataka jer se bavim promocijom iste u okviru organizacije Mozilla Srbija.

%kraj teksta koji se predaje recenzentima
%!!!ne brisite prethodnu liniju!!!

%%%%%%%%%%%%%%%%%%%%%%%%%%%%%%%%%%%%%%%%%%%%%%%%%%%%%%%%%%%%%%%%%%%%%%%%%%%%%%%%%%%%%%%%%%
\section{Poverljivi komentari}
% Poverljivi komentari neće biti prosleđeni autorima seminarskog rada.
% Ukoliko nemate poverljivih komentara, ovaj deo može da ostane prazan.
%%%%%%%%%%%%%%%%%%%%%%%%%%%%%%%%%%%%%%%%%%%%%%%%%%%%%%%%%%%%%%%%%%%%%%%%%%%%%%%%%%%%%%%%%%
Sve mi lepo zvuči, kao da je previše dobro sročeno, ali nikako ne mogu da iščeprkam dodatnu literaturu kako bih uporedio sa nečim. Možda grešim, to me veoma kopka.


\end{document}


